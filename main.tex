\documentclass[12pt,a4paper,hyperref={pdfpagelayout={TwoPageRight}}]{moderncv}        % possible options include font size ('10pt', '11pt' and '12pt'), paper size ('a4paper', 'letterpaper', 'a5paper', 'legalpaper', 'executivepaper' and 'landscape') and font family ('sans' and 'roman')

\moderncvstyle{classic}                             % style options are 'casual' (default), 'classic', 'banking', 'oldstyle' and 'fancy'
\moderncvcolor{blue}                               % color options 'black', 'blue' (default), 'burgundy', 'green', 'grey', 'orange', 'purple' and 'red'
%\renewcommand{\familydefault}{\sfdefault}         % to set the default font; use '\sfdefault' for the default sans serif font, '\rmdefault' for the default roman one, or any tex font name

\usepackage[a4paper,top=1.5cm, bottom=1.5cm, left=1cm, right=1cm]{geometry}
\usepackage{cmap}					% поиск в PDF
\usepackage{mathtext} 				% русские буквы в формулах
\usepackage[T2A]{fontenc}			% кодировка
\usepackage[utf8]{inputenc}			% кодировка исходного текста
\usepackage[english,russian]{babel}	% локализация и переносы

\usepackage{amsmath}
\usepackage{indentfirst}
\usepackage{longtable}
\usepackage{graphicx}
\usepackage{array}

\usepackage{wrapfig}
\usepackage{siunitx} % Required for alignment
\usepackage{multirow}
\usepackage{rotating}
\usepackage{caption}
\usepackage{tabto}

\setlength{\hintscolumnwidth}{0.19 \linewidth}
%\setlength{\makecvtitlenamewidth}{10cm}

\definecolor{urlblue}{RGB}{56, 115, 178}

\def\slink#1#2{\textcolor{urlblue}{\underline{\href{#1}{#2}}}}

\def\hdrinfo#1#2{
    \href{#2}{\texttt{#2}} \hspace*{\fill} \fbox{#1} \vspace*{0.5em}\\
}

\def\hdr#1{\textbf{\underline{#1}}}

\def\vlimiter{\vspace{0.6em}}



\name{Рожков}{Александр}

\title{Студент первого курса ФРКТ МФТИ, 19 лет}

\phone[mobile]{+7 (918)-129-95-95}
\email{ralex2304@gmail.com}
\social[github]{ralex2304}
\social[telegram]{ralex2304}
\extrainfo{\url{https://ralex2304.ru}}
\photo[100pt][0.4pt]{img/portrait.jpg}

\makeatletter\renewcommand*{\bibliographyitemlabel}{\@biblabel{\arabic{enumiv}}}\makeatother

\begin{document}
\makecvtitle

\section{Образование}
\cventry{2023 -- настоящее время}{Бакалавриат прикладная математика и физика}{ФРКТ МФТИ}{1 курс}{}{Средний балл - 8.2/10 $\mid$ Средний балл по программированию - 9.5/10}{}

\section{Проекты}

\subsection{МФТИ, курс Дединского И.Р.}

\cvitem{Май 2024}{\hdr{Компилятор и язык программирования}}{\hdrinfo{C/C++, asm x86-64, make, clangd, perf, gdb, ghidra, dot}{https://github.com/ralex2304/Lang}}{\hspace*{1.2em}}{
    Разработка Тьюринг-полного эзотерического языка программирования и компилятора для архитектур x86-64 и собственного \slink{https://github.com/ralex2304/Processor}{SPU}. Реализованы создание и оптимизации AST и IR (свёртка констант, удаление мёртвого кода). Создан \slink{https://github.com/ralex2304/LangStandard}{стандарт AST и IR}.
}

\vlimiter

\cvitem{Апрель 2024}{\hdr{Profile guided хеш таблица}}{\hdrinfo{C/C++, asm x86-64, make, callgrind, perf, matplotlib}{https://github.com/ralex2304/HashTable}}{\hspace*{1em}}{
    PGO реализация хеш таблицы. Сравнение различных хеш функций. Профилирование (Callgrind и Perf) и оптимизации с использованием ассемблерных вставок и интринсиков.
}

\vlimiter

\cvitem{Апрель 2024}{\hdr{SIMD alpha blending}}{\hdrinfo{C/C++, Intel AVX, make, callgrind, bmp}{https://github.com/ralex2304/AlphaBlend}}{\hspace*{1em}}{
    Наложение изображений в формате bmp с учётом прозрачности. Профилирование программы при помощи Callgrind. Оптимизации с использованием интринсиков векторных инструкций (SIMD).
}

\vlimiter

\cvitem{Март 2024}{\hdr{Множество Мандельброта}}{\hdrinfo{C/C++, sfml, asm x86-64, make, callgrind}{https://github.com/ralex2304/AvxMandelbrot}}{\hspace*{1em}}{
    Построение множества Мандельброта. Профилирование при помощи Callgrind. Сравнение уровней оптимизации компиляторов GCC и clang. Использование AVX инструкций. Визуализация при помощи sfml.
}

\vlimiter

\cvitem{Март 2024}{\hdr{Учебные программы на x86-64 asm}}{\hdrinfo{asm x86-64, nasm, make, gdb, ghidra}{https://github.com/ralex2304/x86AsmCourse}}{\hspace*{1em}}{
    Реализация стандартной функции printf на ассемблере, в том числе поддерживаются числа с плавающей точкой (передача аргументов по System V AMD64). Программы частично написанные на C и на ассемблере (вызов функций, написанных на ассемблере, из C и наоборот).
}

\vlimiter

\cvitem{Март 2024}{\hdr{Взлом ассемблерных программ}}{\hdrinfo{asm x86, tasm, gdb, ghidra, cutter (disasm), C/C++, gtk}{https://github.com/ralex2304/AsmCrack}}{\hspace*{1em}}{
    Взаимный взлом программ (crackme) вместе с \slink{https://github.com/fogInSingularity}{Владимиром Наумовым}. Дизассемблерирование при помощи Ghidra, реверс инжиниринг, эксплуатация уязвимости переполнения буфера. Реализация программы-патчера на C (GUI на gtk).
}

\vlimiter

\cvitem{Октябрь 2023}{\hdr{Эмулятор процессора}}{\hdrinfo{C/C++, sfml, make, doxygen}{https://github.com/ralex2304/Processor}}{\hspace*{1em}}{
    Ассемблер, SPU, дизассемблер. Процессор использует стек для вычислений. Реализованы регистры и оперативная память. Видеопамять отображается при помощи sfml.
}

\subsection{Всероссийская олимпиада школьников по технологии}

\cvitem{Апрель 2022}{\hdr{Система умный дом}}{\hdrinfo{python, flask, html, css, js, jquery, nginx, C++ (Arduino)}{https://ralex2304.ru/vsosh2022.pdf}}{\hspace*{1em}}{
    Модульная система умный дом. Локальный сервер на raspberry pi (backend: Nginx + Python Flask; frontend: html + css + js + JQuery). Модульные устройства. Проектирование и изготовление печатных плат (гравировка, нанесение маски, пайка). Устройства на основе микроконтроллеров Atmel AtMega 328, 128 и Espressif ESP 8266 (программирование в фреймворке Arduino). 3D моделирование и печать корпусных элементов.
}

\section{Навыки}

\cvitem{IT}{C/C++, asm x86-64, sfml, gtk, git, make, gdb, ghidra, dot, perf, valgrind, latex, doxygen, html, python, jupyter notebook, matplotlib, MS/libre office}
\cvitem{Инженерия}{Компас 3Д, Autodesk Fusion 360, EasyEda, оформление документации по ЕСКД}
\cvitem{Soft skills}{Уверенность, усердие, критическое мышление}
\cvitem{Хобби}{Электроника и микроконтроллеры, 3D печать, металлобработка}
\cvitem{Языки}{Английский (B1 - TOEFL + Cambridge PET), Русский (родной)}

\section{Достижения}

\cventry{2021--2023}{Всероссийская олимпиада школьников по технологии}{заключительный этап}{трёхкратный призёр}{}{}

\cventry{2023}{Открытая олимпиада ИТМО по информатике}{}{призёр второй степени}{}{}

\cventry{2023}{Всероссийская олимпиада школьников по информатике}{региональный этап (Краснодарский край)}{победитель}{}{}

\end{document}

